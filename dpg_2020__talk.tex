

\documentclass[aspectratio=169]{beamer}

%###############################################
%### Header ##################################
%###############################################


\usepackage[utf8]{inputenc}
\usepackage[T1]{fontenc}
\usepackage{eso-pic, picture}
\usepackage{pgf}
\usepackage[absolute, overlay]{textpos}





\usetheme{daniels}


%###############################################
%### Presentation ##############################
%###############################################


\begin{document}


% Title Slide
\title{The MonXe Radon Emanation Chamber}
\subtitle{Hardware R\&D Towards DARWIN:}
\date{10th DARWIN Collaboration Meeting\\ $\text{10}^{\text{th}}$ December 2019}
\author{D{\fontsize{10}{10}\selectfont ANIEL} B{\fontsize{10}{10}\selectfont AUR}\\ \texttt{daniel.baur@physik.uni-freiburg.de}} %{Euclid of Alexandria 
\begin{frame}
\titlepage
\end{frame}


% Placeholder Slide
\begin{frame}
    \chapterslide{Is that why I have been so itchy?}
\end{frame}


% Placeholder Slide
\begin{frame}
    \frametitle{Experimental Setup}
    Hier könnte ihre Werbung stehen.
    \framenumber
%\framesubtitle{The proof uses \textit{reductio ad absurdum}.} 
%\begin{theorem}
%There is no largest prime number. \end{theorem} 
%\begin{enumerate} 
%\item<1-| alert@1> Suppose $p$ were the largest prime number. 
%\item<2-> Let $q$ be the product of the first $p$ numbers. 
%\item<3-> Then $q+1$ is not divisible by any of them. 
%\item<1-> But $q + 1$ is greater than $1$, thus divisible by some prime
%number not in the first $p$ numbers.
%\end{enumerate}
\end{frame}


\end{document}






